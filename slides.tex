\documentclass[english,serif,mathserif,usenames,dvipsnames]{beamer}
\usetheme[informal]{gc3}

\usepackage[T1]{fontenc}
\usepackage[utf8]{inputenc}
\usepackage{babel}

%% This is optional: it adds a few commands and environment we
%% regularly use in our slide sets
\usepackage{gc3}

\begin{document}

%% Optional Argument in [Brackets]: Short Title for Footline
\title[Intro to SLURM]{Introduction to batch computing using SLURM}
\author{Riccardo Murri \texttt{<riccardo.murri@uzh.ch>}}
\date{2018-04-17}

%% Makes the title slide
\maketitle

\section{General}
\begin{frame}
  \frametitle{What is a batch-queueing system?}

  \begin{quote}
    A batch-queuing system is a way to execute computational jobs
    \emph{asynchronously}.
  \end{quote}

  \pause \+
  You submit a script to be processed to a central
  resource scheduler, and the scheduler executes the script when
  enough resources (CPUs, memory, disk space, etc.) are available.

  \pause \+
  \emph{Note:} This also means that you cannot
  \emph{interactively} type input to the script! Batch computing is
  mostly intended for \emph{non-interactive} tasks.
\end{frame}

\begin{frame}
  \frametitle{What is SLURM?}

  SLURM is one of many batch-queueing systems available for GNU/Linux
  clusters.  It is the system available on \texttt{cluster.pelkmanslab.org}.

  \+
  The SLURM batch-queuing system can distribute job execution across
  a cluster of independent computing nodes so that many jobs can be
  executed at the same time.

  \+
  \begin{references}
    \url{http://slurm.schedmd.com/}
  \end{references}
\end{frame}


\part{SLURM commands}
\begin{frame}
  \frametitle{How do I interact with SLURM?}

  SLURM provides several commands to control job submission and
  status:
  \begin{itemize}
  \item \texttt{sbatch} -- enqueue a \emph{shell script} for execution
  \item \texttt{squeue} -- display information about running or pending jobs
  \item \texttt{sacct} -- display information about completed jobs
  \item \texttt{srun} -- run command interactively
  \item \texttt{scancel} -- remove job from queue or kill running job
  \end{itemize}

  \+
  All these commands provide a \texttt{man} page full of information.
  (E.g., run \texttt{man sacct})
\end{frame}


\begin{frame}[fragile]
  \frametitle{sbatch}

  You can submit a \emph{shell script} for processing by SLURM using
  the \texttt{sbatch} command like this:

\begin{sh}
  sbatch my_script.sh
\end{sh}

\+
  {\em Using the \texttt{sbatch} command, by default your job will run on 1
  CPU and using the entire memory of a compute node (32GB).}

\+
  Options to \texttt{sbatch} are available to request more CPUs or a
  different memory slice.
\end{frame}


\begin{frame}[fragile]
  \frametitle{Example: run MATLAB command}

  The following script can be used to execute MATLAB commands in a SLURM job:
  \lstinputlisting{examples/run_matlab_cmd.sh}

  \+
  You can submit this MATLAB job to SLURM by running:

\begin{sh}
    # note: it's `hello` not `hello.m`
    sbatch run_matlab_cmd.sh hello
\end{sh}
\end{frame}
% $

\begin{frame}[fragile]
  \frametitle{squeue}
  Once you have submitted a job, you can monitor its status with \texttt{squeue}:

\begin{semiverbatim}\tiny
\$ squeue
JOBID PARTITION     NAME     USER ST       TIME  NODES NODELIST(REASON)
  502      main run_matl   rmurri  R       0:03      1 pelkmanslab-slurm-worker-001
\end{semiverbatim}

  \+
  \emph{Note:} once the job is finished, it won't be listed by
  \texttt{squeue} anymore.

\end{frame}
% $


\begin{frame}[fragile]
  \frametitle{sacct}

  \+
  You can list already-finished jobs with the \texttt{sacct} command:

\begin{semiverbatim}\tiny
\$ sacct
       JobID    JobName  Partition    Account  AllocCPUS      State ExitCode
------------ ---------- ---------- ---------- ---------- ---------- --------
502          run_matla+       main       root          1  COMPLETED      0:0
502.batch         batch                  root          1  COMPLETED      0:0
\end{semiverbatim}

  \+ \emph{Note:} By default, \texttt{sacct} lists only jobs finished
  since last midnight (and until now).  Use options
  \texttt{-{}-starttime} and \texttt{-{}-endtime} to change this
  window.
\end{frame}
% $


\begin{frame}[fragile]
  \frametitle{Where have all the outputs gone?}

  Standard output \emph{and} standard error output of a job are (by
  default) collected in a file named
  \texttt{slurm-}$\mathtt{JJJ}$\texttt{.out}:

  \begin{semiverbatim}\tiny
\$ cat slurm-502.out

                            < M A T L A B (R) >
                  Copyright 1984-2017 The MathWorks, Inc.
                   R2017b (9.3.0.713579) 64-bit (glnxa64)
                             September 14, 2017


To get started, type one of these: helpwin, helpdesk, or demo.
For product information, visit www.mathworks.com.

Hello, SLURM!
\end{semiverbatim}

\+
  This can be changed using options \texttt{-o} and \texttt{-e} to \texttt{sbatch}.
\end{frame}
% $


\begin{frame}[fragile]
  \frametitle{(Common) Options to \texttt{sbatch}}

  You can specify options to \texttt{sbatch} directly \emph{in the job
    script}, as special comments introduced by the \texttt{\#SBATCH}
  prefix:

  \begin{semiverbatim}\footnotesize\ttfamily
 ## use 1 CPU
{\bfseries #SBATCH} -c 1

 ## run for (max) 8 hours
{\bfseries #SBATCH} -{}-time=8:00:00

 ## use (max) 3500 MB of memory per CPU
{\bfseries #SBATCH} -{}-mem-per-cpu=3500m

 ## write both output and errors to file `run_matlab.NNN.log`
{\bfseries #SBATCH} -o run_matlab.\%j.log
{\bfseries #SBATCH} -e run_matlab.\%j.log
  \end{semiverbatim}
\end{frame}


\begin{frame}[fragile]
  \frametitle{(Common) Options to \texttt{sbatch}}

  You can also specify options to \texttt{sbatch} on the
  command line:

  \begin{semiverbatim}
\$ sbatch -{}-time=8:00:00 \textbackslash
          -{}-mem-per-cpu=3500m \textbackslash
          run_matlab_cmd.sh hello
\end{semiverbatim}

  \+ If the same option is given \emph{both} in the script \emph{and}
  on the command line, the command-line wins.
\end{frame}


\begin{frame}[fragile]
  \frametitle{Canceling a job}

  Command \texttt{scancel} removes a job from the queue, or kills it
  immediately if it's already running.

  \+
  You must give \texttt{scancel} the job ID(s) of the job(s) you
  want to abort:

\begin{semiverbatim}
\${\bfseries scancel 502}
\end{semiverbatim}

  \+
  \emph{Note:} \texttt{scancel} gives no output or feedback!
\end{frame}


\begin{frame}[fragile]
  \frametitle{squeue}
  You can check the state of (anyone's) jobs with the \texttt{squeue}
  command:

\begin{sh}
  # view all jobs
  squeue

  # view only jobs of \$USER
  squeue -u $USER

  # view only running jobs (af any user)
  squeue --state RUNNING
\end{sh}

\end{frame}
% $


\begin{frame}[fragile]
  \frametitle{Changing the default output of squeue}

  Environmental variable \texttt{SQUEUE\_FORMAT2} sets the columns that
  are displayed in the table produced by command \texttt{squeue}:

  \begin{semiverbatim}\tiny
\$ export{\bfseries SQUEUE_FORMAT2="jobid:6,username:12,name,state,reason,timeused:12,timeleft:12"}
\$ squeue
JOBID USER        NAME                STATE       REASON              TIME        TIME_LEFT
504   rmurri      sleep.sh            RUNNING     None                0:43        UNLIMITED
\end{semiverbatim}

  \+
  Read \texttt{man squeue} to find out all the possible column names.

  \+ If you don't like the default \texttt{squeue} columns, you can
  (and should!) set a new value for \texttt{SQUEUE\_FORMAT2} in your
  \texttt{.bashrc} file.
\end{frame}


\begin{frame}[fragile]
  \frametitle{Changing the default output of \texttt{sacct}}

  Environmental variable \texttt{SACCT\_FORMAT} sets the columns that
  are displayed in the table produced by command \texttt{sacct}:

  \begin{semiverbatim}\tiny
\$ export{\bfseries SACCT_FORMAT="jobid,user,state,exitcode,start,elapsed"}
\$ sacct
       JobID      User      State ExitCode               Start    Elapsed
-{}-{}-{}-{}-{}-{}-{}-{}-{}-{}-{}- -{}-{}-{}-{}-{}-{}-{}-{}- -{}-{}-{}-{}-{}-{}-{}-{}-{}- -{}-{}-{}-{}-{}-{}-{}- -{}-{}-{}-{}-{}-{}-{}-{}-{}-{}-{}-{}-{}-{}-{}-{}-{}-{}- -{}-{}-{}-{}-{}-{}-{}-{}-{}-
502             rmurri  COMPLETED      0:0 2018-04-16T14:27:33   00:00:13
502.batch               COMPLETED      0:0 2018-04-16T14:27:33   00:00:13
\end{semiverbatim}

  \+
  Use \texttt{sacct -{}-helpformat} to list all possible column names.

  \+ If you don't like the default \texttt{sacct} columns, you can
  (and should!) set a new value for \texttt{SACCT\_FORMAT} in your
  \texttt{.bashrc} file.
\end{frame}


\begin{frame}[fragile]
  \frametitle{Interactive sessions}

  Use command \texttt{srun -{}-pty} to run a job interactively.

  \+
  For instance, to start a shell on a compute node:
\begin{semiverbatim}
\${\bfseries srun --pty bash --login}
\end{semiverbatim}

  \+ \emph{Note:} The \texttt{srun} command will block until an
  execution slot is available.
\end{frame}


\part{Dealing with multiple software versions}

\begin{frame}
  \frametitle{The Bad News}

  \begin{quote}
    There is no generic way of installing (and easily switching)
    multiple versions of the same software or software stack.
  \end{quote}

  \+
  The easy-to-use systems are language- or
  software-specific (e.g., Python).

  \+
  The generic systems revolve around controlling environmental
  variables (but have a lot of caveats and edge cases).
\end{frame}

\begin{frame}[fragile]
  \frametitle{Solution for Python: virtualenv}

  Python has ``virtual environments'', which are just directories
  containing a copy of Python and libraries.

  \+
  Once you \emph{``activate''} a virtualenv, every time you install
  a Python library (\texttt{pip install}), it is installed in the
  virtualenv.

  \+
  You can delete a virtualenv by simply removing its directory:
\begin{semiverbatim}
\$ rm -r my_python_virtualenv
\end{semiverbatim}
\end{frame}


\begin{frame}
  \frametitle{Using Python's virtualenv}
  \small

  \begin{itemize}
  \item \textbf{Create a virtualenv} (once only)

\begin{semiverbatim}
\$ python -m virtualenv /path/to/venv/
\end{semiverbatim}

    \emph{Note:}
    \begin{itemize}
    \item The virtualenv directory \emph{must not} exist!
    \item Any filesystem path is fine, but choose one on a shared
      filesystem (e.g. your home directory) if you want to use it in
      jobs across the cluster.
    \end{itemize}

  \item \textbf{Enter (``activate'') a virtualenv} (in every new shell)

\begin{semiverbatim}
\$ source /path/to/venv/bin/activate
\end{semiverbatim}

  \item \textbf{Exit a virtualenv}

\begin{semiverbatim}
\$ deactivate
\end{semiverbatim}

  \end{itemize}
\end{frame}


\begin{frame}[fragile]
  \frametitle{Using virtualenvs in SLURM jobs}

  Commands in a job script are no different from the commands you type at the shell prompt, so just activate the virtualenv.
  \begin{sh}
# `.` is an alias for `source` that works always
venv="$HOME/path/to/virtualenv"
. "$venv/bin/activate"

# `exec` must be the very last statement
exec python my_script.py
  \end{sh}
\end{frame}
% $


\begin{frame}
  \frametitle{The generic solution: \em environment~modules}

  The \texttt{module} command is the standard solution on HPC clusters
  for managing multiple versions of the same software.

  \+ In essence, just manipulates the shell environment, adding or
  removing variables, or changing the value of the existing ones
  (e.g., add prepend directories to \texttt{PATH}).  All the changes
  are listed in a single (Lua-syntax) file, and are performed (or
  undone) at the same time.
\end{frame}


\begin{frame}[fragile]
  \frametitle{Listing available modules}
  \small

  You can list available module files with the command \texttt{module
    avail} (can be abbreviated to \texttt{ml av}):

\begin{semiverbatim}\tiny\ttfamily
$ \textbf{ml av}

-{}-{}-{}-{}-{}-{}-{}-{}-{}-{}-{}-{}-{}-{}-{}-{}-{}-{}-{}-{}-{}-{}-{}-{}-{}-{} /opt/modulefiles -{}-{}-{}-{}-{}-{}-{}-{}-{}-{}-{}-{}-{}-{}-{}-{}-{}-{}-{}-{}-{}-{}-{}-{}-{}-{}-{}-{}-{}
   MATLAB/R2013b    MATLAB/R2017b (D)

-{}-{}-{}-{}-{}-{}-{}-{} /opt/spack/share/spack/modules/linux-{}ubuntu16.04-{}x86_64 -{}-{}-{}-{}-{}-{}-{}-{}
   intel-parallel-studio-professional.2018.0-gcc-5.4.0-qx4ndcf

-{}-{}-{}-{}-{}-{}-{}-{}-{}-{}-{}-{}-{}-{}-{}-{}-{}-{}-{}-{}-{} /opt/easybuild/modules/all -{}-{}-{}-{}-{}-{}-{}-{}-{}-{}-{}-{}-{}-{}-{}-{}-{}-{}-{}-{}-{}-{}-{}-{}
   EasyBuild/3.5.2    EasyBuild/3.5.3 (D)

-{}-{}-{}-{}-{}-{}-{}-{}-{}-{}-{}-{}-{}-{}-{}-{}-{}- /opt/lmod/lmod/modulefiles/Core -{}-{}-{}-{}-{}-{}-{}-{}-{}-{}-{}-{}-{}-{}-{}-{}-{}-{}-{}-{}-{}-
   lmod/7.0    settarg/7.0

  Where:
   D:  Default Module
   {\em [\ldots]}
\end{semiverbatim}

  \+
  Note that each module name consists of a ``software name'' and a ``version'',
  separated by a slash ``\texttt{/}''.
\end{frame}


\begin{frame}[fragile]
  \frametitle{Loading modules}
  \small

  The command \texttt{module load} (short: \texttt{ml}) performs all
  the changes to the environment described in a module file, and
  ``activates'' a specific version of software:

\begin{semiverbatim}\scriptsize
  ${\bfseries ml MATLAB/R2013b}

  $ matlab -nodisplay -nodesktop -r version
 {\em [\ldots]}
ans =

8.2.0.701 (R2013b)
\end{semiverbatim}

  \+ Loading another version of the same module unloads the first one
  automatically:
\begin{semiverbatim}\scriptsize
  ${\bfseries ml MATLAB/R2017b}

The following have been reloaded with a version change:
  1) MATLAB/R2013b => MATLAB/R2017b

\end{semiverbatim}
\end{frame}

\begin{frame}[fragile]
  \frametitle{Example: MATLAB module file}
  \small

  Command \texttt{module show} displays the contents of a module file:
\begin{semiverbatim}\tiny\ttfamily
\${\bfseries ml show MATLAB/R2013b}
-{}-{}-{}-{}-{}-{}-{}-{}-{}-{}-{}-{}-{}-{}-{}-{}-{}-{}-{}-{}-{}-{}-{}-{}-{}-{}-{}-{}-{}-{}-{}-{}-{}-{}-{}-{}-{}-{}-{}-{}-{}-{}-{}-{}-{}-{}-{}-{}-{}-{}-{}-{}-{}-{}-{}-{}-{}-{}-{}-{}-{}-{}-{}-{}-{}-{}-{}-{}-{}-{}-{}-{}-{}-{}-{}-{}-{}-{}-{}-{}
   /opt/modulefiles/MATLAB/R2013b.lua:
-{}-{}-{}-{}-{}-{}-{}-{}-{}-{}-{}-{}-{}-{}-{}-{}-{}-{}-{}-{}-{}-{}-{}-{}-{}-{}-{}-{}-{}-{}-{}-{}-{}-{}-{}-{}-{}-{}-{}-{}-{}-{}-{}-{}-{}-{}-{}-{}-{}-{}-{}-{}-{}-{}-{}-{}-{}-{}-{}-{}-{}-{}-{}-{}-{}-{}-{}-{}-{}-{}-{}-{}-{}-{}-{}-{}-{}-{}-{}-{}
help([[
  This module loads MATLAB path and environment variables.
]])
prepend_path("PATH","/opt/MATLAB/R2013b/bin")
\end{semiverbatim}

  \+ Here, \texttt{MATLAB/R2013b} is the module for software
  \texttt{MATLAB}, version \texttt{R2013b}, stored in file
  \texttt{MATLAB/R2013b.lua}.

  \+
  Complete list of functions that can be used in a Lua module file:
  \url{http://lmod.readthedocs.io/en/latest/050_lua_modulefiles.html}
\end{frame}


\begin{frame}
  \frametitle{Where to put my own module files?}

  The command \texttt{module use} adds a directory to the search path
  for module files.

  \+ The usual choice is to store one's own module files into
  \texttt{\$HOME/modulefiles}.

  \+ You can add the \texttt{module use} command to your
  \texttt{.bashrc} file to automatically add a module directory each
  time you start a new shell.
\end{frame}


\begin{frame}[fragile]
  \frametitle{How to use \texttt{module} in shell scripts}

  The \texttt{module} command is loaded by file
  \texttt{/etc/profile.d/modules.sh} which you need to \emph{source}:

\begin{sh}
  #! /bin/bash

  # load the `module` command
  . "/etc/profile.d/modules.sh"

  # load an older version of MATLAB
  module load MATLAB/R2013b

  # run it
  exec matlab -nodesktop -nodisplay -r version
\end{sh}
\end{frame}


\part{Job arrays}

\begin{frame}[fragile]
  \frametitle{Job arrays}

  A \emph{job array} is a $1D$ collection of jobs; each job is
  distinguished by its \emph{index} in the array (an integer number).

  \+ Job arrays display as a single row in \texttt{squeue} output, but
  effectively many jobs will be run.

  \+
  Job arrays are created using the \texttt{-{}-array} option to \texttt{sbatch}:
  \begin{sh}
  # submit an array of 12 jobs
  sbatch --array 0-11 array.sh

  # submit 100 jobs, but only allow 10
  # to run at the same time
  sbatch --array '0-99%10' array.sh
  \end{sh}

\end{frame}


\begin{frame}[fragile]
  \frametitle{Example: Job array script}
  \small

  A job array script is like any job script, but you should use the
  \texttt{\$SLURM\_ARRAY\_TASK\_ID} environment variable to detect the
  task to run.

  \begin{sh}
# note: `case ... in` uses *string* comparison!
case "$SLURM_ARRAY_TASK_ID" in

    "1")
        param1=1.5
        param2=3.4
        exec matlab -r simul $param1 $param2
        ;;

    "2")
      # \ldots

esac
  \end{sh}
\end{frame}
% $


\part{The End}
\begin{frame}[fragile]
  \centering

  \begin{center}
    \begin{Huge}
      Thank you!
    \end{Huge}

    \+
    \begin{Large}
      Any questions?
    \end{Large}
  \end{center}
\end{frame}

\begin{frame}[fragile]
  \centering

  \begin{center}
    \begin{Large}
      Thank you!
    \end{Large}

    \+
    \begin{Huge}
      Any questions?
    \end{Huge}
  \end{center}
\end{frame}

\end{document}


%%% Local Variables:
%%% mode: latex
%%% TeX-master: t
%%% End:
